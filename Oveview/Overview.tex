\documentclass{article}

\usepackage{amsmath}
\usepackage{amssymb}
\usepackage[hidelinks]{hyperref}
\usepackage
[
        a4paper,
        hmargin=3cm,
        vmargin=2cm,
]
{geometry}

\title{Loopless stochastic methods. Overview}
\date{Fall 2023}
\author{
  Dmitry Beresnev, d.beresnev@innopolis.university
  \and
  Vsevolod Klyushev, v.klyushev@innopolis.university
}

\begin{document}
\maketitle
\section{L-SVRG and L-Katyusha}

This section is devoted to the article
\href{https://proceedings.mlr.press/v117/kovalev20a.html}
{"Don`t Jump Through Hoops and Remove Those Loops:
  SVRG and Katyusha are Better Without the Outer Loop"}.

\subsection{Background}


Authors deal with the \textbf{empirical risk minimization (finite-sum) problems},
which have the following form:
\[  \min_{x \in \mathbb{R}^d } f(x) \equiv \frac{1}{n} \sum_{i=1}^{n} f_i(x)
\tag{1}\label{1} \]

Finite-sum problems form the \textbf{dominant paradigm for training the supervised
machine learning models}: \( f_i(x) \) is the loss of the model \(x\) on data point \(i\).

The most remarkable algorithms for solving \eqref{1} are
\textit{variance-reduced} stochastic gradients algorithms, which are significantly
faster than SGD on (strongly) convex. However, these methods are not quite
successful in general non-convex problems.

Two of the most notable variance-reduced algorithms are \textbf{SVRG}
(stochastic variance-reduced gradient) method and its accelerated variant
known as \textbf{Katyusha}. Katyusha accelerates SVRG via employment
"negative momentum" idea. Both methods have \textbf{a double loop design}:
at the beginning of the outer loop, , a full pass over the training data is made
to compute \textbf{the total gradient} of \(f\) at r eference point \(w_k\).
In SVRG the reference point is chosen as the freshest iterate, and in Katyusha
\(w_k\) is a weighted average of recent iterates. The total gradient is then
used in the inner loop to \textit{adjust} the stochastic gradient
\(\nabla f_i(x^k)\) (\(x^k\) is the current iterate). In particular, both methods
perform the following adjustment:
\[g^k=\nabla f_i(x^k)-\nabla f_i(w^k)+\nabla f(w^k). \tag{2}\label{2}\]
It turns out that as methods progress, the variance of \(g^k\) progressively
decreases to zero, what effects in \textbf{significantly faster convergence}.

Under assumptions of \(L\)-smoothness and \(\mu\)-strongly convexity
of \(f\), the iterations complexities are the followings:
\begin{itemize}
  \item \(\mathcal{O}((n+\frac{L}{\mu})\log \frac{1}{\epsilon} )\) for SVRG
  \item \(\mathcal{O}((n+\sqrt{\frac{nL}{\mu}})\log \frac{1}{\epsilon} )\)
  (accelerated) for Katyusha
\end{itemize}
what is vast improvement on linear rate of GD and sublinear rate of SGD.

\subsection{Problem statement}
As explained below, the key trade-mark structural feature of SVRG and Katyusha
is presence of the outer loop where a pass over full data is made.
However, the outer loop causes such problems as
\begin{itemize}
  \item The methods are hard to analyze
  \item One needs to decide at which point to terminate the inner loop and
  start the outer loop
\end{itemize}
Elaborating the second issue, the theoretically optimal inner loop size for SVRG
depends on both \(L\) and \(\mu\). However, \(\mu\) is not always known, and
even if estimate is available, it can be very loose. Due to these issues,
inner loop size is often chosen in a suboptimal way.

\subsection{Main idea}
Authors address the above issues by developing \textbf{loopless} variant
of both SVRG and Katyusha: L-SVRG and L-Katyusha respectively. In these methods,
authors remove the outer loop and replace it with a \textit{biased coin-flip},
which on every iteration decides if the gradient \(\nabla f(w^k)\) should be
calculated. In particular, at each step the following happens:
\begin{itemize}
  \item With small probability \(p > 0\), the full pass over data is performed and
  the reference gradient \(\nabla f(w^k)\) is updated
  \item With probability \(1-p\), the previous reference gradient is kept.
\end{itemize}
This procedure can also be interpreted as \textbf{having an outer loop of
random length}.

\subsection{Results}
The paper demonstrates that both proposed methods,
L-SVRG and L-Katyusha, have the following advantages:
\begin{enumerate}
  \item Loopless methods are  \textbf{easier to write down and analyze}
  than original double loop implementations
  \item Loopless methods have the \textbf{same theoretical convergence rates}
  \item Loopless methods are  \textbf{superior to their loopy variants}
  \item L-SVRG is extremely robust to the choice of \(p\) within the
  theoretical optimal interval
\end{enumerate}

The first point follows directly from the article: all the necessary proofs,
lemmas and theorems are quite simple and compact. Moreover, most other
theoretical results, as optimal interval of \(p\) for L-SVRG, are not
sophisticated either.

The second point is proved inside the article with the help of several lemmas and
a couple of theorems.

The third and fourth points are demonstrated through numerical experiments.
In particular, authors show the following results:
\begin{itemize}
  \item \textit{The performance of L-Katyusha is at least as good as
  that of Katyusha, and can be significantly faster in some causes}
  \item \textit{Even the worst case for L-SVRG outperforms the best case for SVRG}
\end{itemize}


\end{document}
